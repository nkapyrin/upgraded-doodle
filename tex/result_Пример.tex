\documentclass[12pt]{article}   % list options between brackets
\usepackage[T2A]{fontenc}
\usepackage[utf8]{inputenc}
\usepackage[english, russian]{babel}	% Языки: русский, английский

\usepackage{anysize,epstopdf,float,caption,enumitem,longtable}              

%\marginsize{left}{right}{top}{bottom}:
\marginsize{3cm}{2cm}{2cm}{2cm}
\usepackage{marginnote}

\usepackage{tikz,circuitikz,amsmath}
\usepackage{qrcode}
\usetikzlibrary{arrows,shapes}

\usepackage{array,multirow} % позволяет выравнивать текст в ячейках по вертикали


\usepackage{listings} % Включение кусков кода
\lstset{language=Python, inputencoding=cp1251, breaklines=true, frame=single, showstringspaces=false, basicstyle=\ttfamily}
\lstset{language=HTML, inputencoding=cp1251, breaklines=true, frame=single, showstringspaces=false, basicstyle=\ttfamily}



\renewcommand{\thefigure}{\arabic{figure}}
\renewcommand{\thetable}{\arabic{table}}
\graphicspath{{./img/}}
\usepackage{hyperref}
\usepackage[hypcap]{caption}

% Убрать коробки вокруг гиперссылок
\hypersetup{
  colorlinks   = true, %Colours links instead of ugly boxes
  urlcolor     = red, %Colour for external hyperlinks
  linkcolor    = red, %Colour of internal links
  citecolor   = red %Colour of citations
}

\usepackage{subcaption}

\relpenalty=9999
\binoppenalty=9999

\DeclareCaptionLabelSeparator{tb}{:\\}

\usepackage{tabularx,ragged2e,booktabs,caption}
\newcolumntype{C}[1]{>{\Centering}m{#1}}
\renewcommand\tabularxcolumn[1]{C{#1}}
\captionsetup[table]{justification=raggedleft,
  singlelinecheck=false,labelsep=tb
}

%\usepackage{chngcntr}
%\counterwithin{table}{chapter}
%\counterwithin{figure}{chapter}

\makeatletter
  \def\thesubfigure{\textit{\asbuk{subfigure}}}
  \providecommand\thefigsubsep{}
  \def\p@subfigure{\@nameuse{thefigure}\thefigsubsep}
\makeatother

\newcolumntype{P}[1]{>{\centering\arraybackslash}p{#1}}
\newcolumntype{M}[1]{>{\centering\arraybackslash}m{#1}}
\newcommand*\rot{\rotatebox{90}}

%%% Библиография %%%

\usepackage{cite} % Красивые ссылки на литературу
\makeatletter
\bibliographystyle{bib/utf8gost705u}	% Оформляем библиографию согласно ГОСТ 7.0.5
\renewcommand{\@biblabel}[1]{#1.}	% Заменяем библиографию с квадратных скобок на точку:
\makeatother
\addto{\captionsrussian}{\renewcommand{\bibname}{Список использованных источников}}


%%% Список сокращений %%%
\usepackage[acronym,nopostdot,nonumberlist]{glossaries}
\makeglossaries


\renewcommand{\glossarysection}[2][]{}
%\setlength{\glssymbolwidth}{.1\hsize}
%\setlength{\glsdescwidth}{15cm}





		% Подключаемые пакеты

\usepackage{ifthen}

\begin{document}









\thispagestyle{empty}
\begin{center}
%\begin{center}\includegraphics[width=1cm]{mai.pdf}\end{center}\bigskip

{\scriptsize
МИНИСТЕРСТВО ОБРАЗОВАНИЯ И НАУКИ РОССИЙСКОЙ ФЕДЕРАЦИИ\\[.2cm]

ФЕДЕРАЛЬНОЕ ГОСУДАРСТВЕННОЕ БЮДЖЕТНОЕ ОБРАЗОВАТЕЛЬНОЕ\\[.2cm]

УЧРЕЖДЕНИЕ ВЫСШЕГО ОБРАЗОВАНИЯ\\[.2cm]

\textbf{<<МОСКОВСКИЙ АВИАЦИОННЫЙ ИНСТИТУТ}\\[.2cm]

\textbf{(НАЦИОНАЛЬНЫЙ ИССЛЕДОВАТЕЛЬСКИЙ УНИВЕРСИТЕТ)>> (МАИ)}\\[.2cm]
}

\vfill
{\large
Отчёт о выполнении практического задания\\
по курсу\\
\textbf{<<Метрология, Стандартизация, Сертификация>>}\\[0.4cm]
\bfseries Оценка влияния погрешности входного сопротивления на вольтметр в сценарии ограничения напряжения между 3 и 10 В\\[2cm]}

\vfill

\begin{minipage}[t]{0.3\textwidth}
	\begin{flushleft}
    Отчёт и материалы:\\\bigskip
	\qrcode{https://www.dropbox.com/sh/x5iftzljl6j9etv/AAAWuwOUTJjTzQEqLyZALQNMa?dl=0}%
    \end{flushleft}
\end{minipage}%
\begin{minipage}[t]{0.7\textwidth}
    \begin{flushright}
	студент студент\\[.4cm]
	\rule{4cm}{1pt} Пример Автора Отчёта\\
	% \rule{4cm}{1pt} Фамилия И.О.\\
	<<\rule{.8cm}{1pt}>> \rule{3cm}{1pt} 2017 г.\\\bigskip
	Принимал работу:\\
	Ст.преп. каф. 303\\[.4cm]
	\rule{4cm}{1pt} Капырин Н.И.\\
	<<\rule{.8cm}{1pt}>> \rule{3cm}{1pt} 2017 г.
    \end{flushright}
\end{minipage}

\vfill

{\large Москва, 2017 г.}
\end{center}














\clearpage

\renewcommand{\baselinestretch}{1.2}
\tableofcontents
\renewcommand{\baselinestretch}{1.5}
%\clearpage

% =====================================================
% Список обозначений и сокращений:
% \section*{Список сокращений} %\printglossaries
% \renewcommand{\baselinestretch}{1.2}
% \addcontentsline{toc}{section}{Список сокращений}	% Добавляем его в оглавление 
% \printglossary[type=\acronymtype] % prints just the list of acronyms
% \renewcommand{\baselinestretch}{1.5}

% =====================================================
% некоторые общеупотребимые сокращения
%\newglossaryentry{BD}{type=\acronymtype,name={БД},description={База данных}}
%\newglossaryentry{LA}{type=\acronymtype,name={ЛА},description={Летательный аппарат}}
% \newglossaryentry{GOST}{type=\acronymtype,name={ГОСТ},description={Государственный Стандарт}}
% \clearpage



\section{Задание}

Оценить влияние погрешности входного сопротивления на вольтметр в сценарии ограничения напряжения между 3 и 10 В.

\section{Описание задачи}

Вольтметр — измерительный прибор непосредственного отсчёта для определения напряжения или ЭДС в электрических цепях. Подключается параллельно нагрузке или источнику электрической энергии. Идеальный вольтметр должен обладать бесконечно большим внутренним сопротивлением. Поэтому чем выше внутреннее сопротивление в реальном вольтметре, тем меньше влияния оказывает прибор на измеряемый объект и, следовательно, тем выше точность и разнообразнее области применения.

При изготовлении вольтметра, входное сопротивление подбирается из условия соблюдения класса точности прибора. Здесь используется линейная модель погрешности изготовления, учитывающая масштабный коэффициент и смещение.

При воздействии погрешности входного сопротивления на вольтметр возникает погрешности измерения, охарактеризованная исходя из математических моделей, приведённых в следующем разделе.

\clearpage





\section{Математические модели}

\subsection{Модель вольтметра}

Модель вольтметра задана следующим уравнением (зависимость индикации вольтметра U\_out от входного напряжения U\_in, масштаба шкалы k\_ind и погрешности связанной с входным сопротивлением R\_inp):

\begin{equation}
    U_{out} = \frac{U_{inp} k_{ind}}{R_{inp}}
\end{equation}


\subsection{Модель погрешности входного сопротивления}

Модель погрешности входного сопротивления задана следующим уравнением (зависимость входного сопротивления от теоретического значения R, масштабного коэффициента a и коэффициента смещения b):

\begin{equation}
    R_{inp} = R a + b
\end{equation}

При взаимодействии, модели явлений, описанных в предыдущих разделах, приводят к модели погрешностей, приведенной в следующем разделе.



\subsection{Модель погрешности}

Модель погрешности измерений, полученная путём взятия градиента от модели уравнений, приведённых в прошлом разделе.

\begin{equation}
    \Delta_{U out} = - \frac{\Delta_{R} U_{inp} a k_{ind}}{\left(R a + b\right)^{2}} - \frac{\Delta_{a} R U_{inp} k_{ind}}{\left(R a + b\right)^{2}} - \frac{\Delta_{b} U_{inp} k_{ind}}{\left(R a + b\right)^{2}}
\end{equation}




\clearpage\thispagestyle{empty}

\begin{center}
\includegraphics[width=.4\textwidth]{qr_tmp/db_qr_0.png}\\\includegraphics[width=.4\textwidth]{qr_tmp/db_qr_1.png}\\\includegraphics[width=.4\textwidth]{qr_tmp/db_qr_2.png}\\
\end{center}

% =====================================================
% библиография внтури текста
%\begin{thebibliography}{9}
%
%\bibitem{latex_book} Leslie Lamport, \emph{\LaTeX: a document preparation system}, Addison Wesley, Massachusetts, 2nd edition, 1994.
%
%\bibitem{how2bib} Оформление библиографического списка [Электронный ресурс]. -- СПб.: ВШМ СПбГУ, 1993 - . -- Режим доступа : \url{http://gsom.spbu.ru/files/upload/library/list_of_literature.pdf}, свободный. Загл. с экрана
%
%\bibitem{laser} \url{https://ru.wikipedia.org/wiki/%D0%9B%D0%B0%D0%B7%D0%B5%D1%80%D0%BD%D1%8B%D0%B9_%D0%B4%D0%B0%D0%BB%D1%8C%D0%BD%D0%BE%D0%BC%D0%B5%D1%80}
%\end{thebibliography}

%\phantomsection
%\addcontentsline{toc}{section}{\bibname}	% Добавляем список литературы в оглавление
%\renewcommand{\baselinestretch}{1.2}
%
%\renewcommand{\baselinestretch}{1.5}


\end{document}
